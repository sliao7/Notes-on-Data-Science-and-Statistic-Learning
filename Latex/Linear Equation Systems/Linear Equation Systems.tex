\documentclass[twoside,12pt]{article}
\usepackage{amsmath,amsfonts,amsthm,fullpage,amssymb}
\usepackage{algorithm,hyperref}
\usepackage{algorithmic}
\usepackage{graphicx}
\usepackage{tcolorbox}
\usepackage{hyperref}
\DeclareMathOperator*{\argmax}{arg\,max}
\DeclareMathOperator*{\argmin}{arg\,min}

\begin{document}
\title{Notes on Linear Systems}
\author {Shasha Liao \\ Georgia Tech}
\maketitle
We often face some linear algebra related problems in data science. Here are a collection of my notes. 
\section{General Cases}
Usually, the problem will be reduced to solve a linear system: 
\begin{align}\label{eq}
Ax = b, \text{ with } A \in \mathbb{R}^{m \times n}, b \in \mathbb{R}^m,
\end{align}
where $m$ is the number of linear equations and $n$ is the number of variables.
There are several cases:
\begin{itemize}
\item Case 1: $A$ is full-ranked
\begin{itemize}
\item Case 1.1: If $A$ is square, then \eqref{eq} has a unique solution $x = A^{-1}b$.
\item Case 1.2: If $A$ is not square, then \eqref{eq} is not solvable. Instead, we consider the least square problem 
\begin{align}\label{ls}
\hat{x} = \argmax_{x \in \mathbb{R}^n} \|Ax - b\|_2.
\end{align}
In this case, the least square problem \eqref{ls} has a unique solution and can be solved by solving the equivalent normal equation $$A^T A \hat{x} = A^T b.$$ Since $A$ is full-ranked, $A^TA$ is invertible. So $$\hat{x} = (A^TA)^{-1}A^Tb.$$
\end{itemize}
\item Case 2: $A$ is not full-ranked.
In this case, no matter $A$ is square or not, we can find the solution to the least square problem by using the psudoinverse: $$\hat{x} = A^{+} b = \argmax_{x \in \mathbb{R}^n} \|Ax - b\|_2 $$ where $A^{+}$ is defined using the SVD of $A$ in the following way:
If $A = U \begin{pmatrix}  \Sigma_r & 0_{12} \\ 0_{21} & 0_{22} \end{pmatrix}  V^T,$ then $A^{+} = V \begin{pmatrix}  \Sigma_r^{-1} & 0_{21} \\ 0_{12} & 0_{22} \end{pmatrix}U^T$. Note that since $A$ is not full-ranked, the least square problem have finite many solutions. 
\end{itemize}

\end{document}